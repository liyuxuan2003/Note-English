% !TEX program = xelatex
\documentclass[UTF8]{ctexart}

\RequirePackage{inputenc}
\RequirePackage{fontspec}
\RequirePackage{xeCJK}

\RequirePackage{subfigure}

\RequirePackage{enumitem}

\setmainfont{Times New Roman}

\setCJKmainfont{等线}
\setCJKsansfont{等线}
\setCJKmonofont{等线}

\RequirePackage{geometry}
\geometry
{
    left=1.25in,
    right=1.25in,
    top=1in,
    bottom=1in
}

\title{英语笔记}
\author{李宇轩}
\date{2020.06.27}

\begin{document}

\maketitle

\newpage

\tableofcontents

\newpage

\setlength{\parindent}{0pt}

\section{词类和句子成分}

\subsection{词的分类}
    词根据其特点可以分为十类:\vspace{8pt}
    \begin{table}[h!]
        \begin{center}
            \begin{tabular}{p{60pt}|p{80pt}|p{80pt}|p{100pt}}
                \hline
                词类&英语名称&英语缩写&示例\\ \hline
                名词&noun&\textit{n.}&apple~~~~banana\\ \hline
                代词&pronoun&\textit{pron.}&he~~~~she~~~~some\\ \hline
                动词&verb&\textit{v.}&run~~~~swim\\ \hline
                形容词&adjective&\textit{adj.}&bad~~~~happy\\ \hline
                副词&adverb&\textit{adv.}&badly~~~~happily\\ \hline
                数词&numeral&\textit{num.}&one~~~~two~~~~third\\ \hline
                冠词&article&\textit{art.}&a~~~~an~~~~the\\ \hline
                介词&preposition&\textit{prep.}&in~~~~on~~~~with\\ \hline
                连词&conjunction&\textit{conj.}&and~~~~or~~~~but\\ \hline
                感叹词&interjection&\textit{int.}&oh~~~~ah\\ \hline
            \end{tabular}
            \caption{词的分类}
        \end{center}
    \end{table}\\
    这种分类称为词类(parts of speech)。\\[3mm]
    实词可以在句中独立充当成分,实词包含了:名词,代词,动词,形容词,副词,数词。\\[3mm]
    虚词不能在句中独立充当成分,虚词包含了:冠词,介词,连词,感叹词。\\[8mm]
    同一个词在不同的场合可以用作不同词类:\\[4mm]
    lead作动词:They \textbf{lead} us through the dark forest.\\[3mm]
    lead作名词:She gained the \textbf{lead} in the race.\\[5mm]
    behind作介词:The sun sank \textbf{behind} the hills.\\[3mm]
    behind作副词:The dog ran \textbf{behind} me.\\[5mm]
    light作形容词:The feather is very \textbf{light}.\\[3mm]
    light作名词:Red \textbf{light} spread above the hills.


\newpage

\subsection{句子种类}
    句子是具有一定的语法结构,表达一个独立完整意义的语言单位。\\[2mm]
    按照句子结构,句子总共可以分为三类:
    \begin{table}[h!]
        \begin{center}
            \begin{tabular}{p{60pt}|p{120pt}}
                \hline
                简单句&simple sentence\\ \hline
                并列句&compound sentence\\ \hline
                复合句&complex sentence\\ \hline
            \end{tabular}
            \caption{句子种类(按句子结构分类)}
        \end{center}
    \end{table}\\
    按照句子功能,句子总共可以分为四类:
    \begin{table}[h!]
        \begin{center}
            \begin{tabular}{p{60pt}|p{120pt}}
                \hline
                陈述句&declarative sentence\\ \hline
                疑问句&interrogative sentence\\ \hline
                祈使句&imperative sentence\\ \hline
                感叹句&exclamatory sentence\\ \hline
            \end{tabular}
            \caption{句子种类(按句子功能分类)}
        \end{center}
    \end{table}

\subsection{句子成分}
    句子成分指的是在句子中承担一定功能的一个部分。\\[3mm]
    句子成分根据其功能可以分为八类:\vspace{5pt}
    \begin{table}[h!]
        \begin{center}
            \begin{tabular}{p{60pt}|p{80pt}|p{180pt}}
                \hline
                句子成分&英语名称&功能\\ \hline
                主语&subject&说明句子主体\\ \hline
                谓语&predicate&说明句子主体的动作内容\\ \hline
                宾语&object&说明句子主体的动作对象\\ \hline
                定语&attributive&修饰名词~~修饰代词\\ \hline
                状语&adverial&修饰动词~~修饰副词~~修饰形容词\\ \hline
                补语&adverial&补充说明主语和宾语\\ \hline
            \end{tabular}
            \caption{句子成分}
        \end{center}
    \end{table}\\
    句子成分,既可以由一定的词或词组充当,也可以由一定的分句充当。

\newpage

\subsubsection{主语}
    主语是说明\textbf{句子主体}的句子成分。\\[3mm]
    主语通常可以由以下内容充当:名词,代词,数词,非谓语动词,名词性从句。\\[6mm]
    主语由名词充当:\textbf{Apple} is sweet.\\[3mm]
    主语由名词充当:\textbf{Rey} said goodbye to Lynn.\\[3mm]
    主语由代词充当:\textbf{She} said goodbye to Lynn.\\[3mm]
    主语由数词充当:\textbf{Eight} is a lucky number.\\[3mm]
    主语由非谓语动词充当:\textbf{Shopping} is important for many people.\\[3mm]
    主语由名词性从句充当:\textbf{What he said} is very important.\\

\subsubsection{谓语}
    谓语是说明\textbf{句子主体动作内容}的句子成分。\\[3mm]
    谓语通常可以由以下内容充当:谓语动词。\\[6mm]
    谓语由谓语动词充当:Apple \textbf{is} sweet.\\[3mm]
    谓语由谓语动词充当:Kevin \textbf{was} a young boy.\\[3mm]
    谓语由谓语动词充当:He \textbf{sends} a email.\\[3mm]
    谓语由谓语动词充当:He \textbf{will do} his homework.\\

\subsubsection{宾语}
    宾语是说明\textbf{句子主体动作对象}的句子成分。\\[3mm]
    宾语通常可以由以下内容充当:名词,代词,非谓语动词,名词性从句。\\[6mm]
    宾语由名词充当:They bought a \textbf{necklace} for my birthday.\\[3mm]
    宾语由代词充当:They answered \textbf{me} without hesitation.\\[3mm]
    宾语由非谓语动词充当:He refused \textbf{doing his homework}.\\[3mm]
    宾语由名词性从句充当:He expressed \textbf{what he felt about it}.

\newpage
    
\subsubsection{定语}
    定语是修饰句子中\textbf{名词}或\textbf{代词}的句子成分。\\[3mm]
    定语通常可以由以下内容充当:形容词,副词,非谓语动词,介词词组,定语从句。\\[6mm]
    定语由形容词充当(修饰名词):This is a very \textbf{sweet} apple.\\[3mm]
    定语由形容词充当(修饰代词):She is a very \textbf{cute} girl.\\[3mm]
    定语由副词充当:I will go to the village \textbf{below}.\\[3mm]
    定语由介词词组充当:Pencil \textbf{inside the desk} is mine.\\[3mm]
    定语由介词词组充当:Book \textbf{on the table} is hers.\\[3mm]
    定语由定语从句充当:He is the man \textbf{who saved the world}.\\[3mm]
    定语由非谓语动词充当:The boy \textbf{playing} basketball is my friend.\\

\subsubsection{状语}
    状语是修饰句子中\textbf{动词}或\textbf{副词}或\textbf{形容词}的句子成分。\\[3mm]
    状语通常可以由以下内容充当:副词,非谓语动词,介词词组,状语从句。\\[6mm]
    状语由副词充当(修饰形容词):This is a \textbf{very} sweet apple.\\[3mm]
    状语由副词充当(修饰副词):He runs \textbf{very} fast.\\[3mm]
    状语由副词充当(修饰动词):He runs very \textbf{fast}.\\[3mm]
    状语由介词词组充当:She swims \textbf{in the swimming pool}.\\[3mm]
    状语由介词词组充当:She plays piano \textbf{in the morning}.\\[3mm]
    状语由状语从句充当:He is talking \textbf{when his teacher comes in}.\\[3mm]
    状语由非谓语动词充当:They cheated \textbf{to get} a higer scores.

\newpage

\subsubsection{补语}
    补语是\textbf{补充说明主语和宾语}的句子成分。\\[3mm]
    补语通常可以由以下内容充当:名词,形容词,副词,介词词组,非谓语动词,名词性从句。\\[3mm]
    补语中对于补充说明主语的,称为主语补语,简称主补。\\[3mm]
    补语中对于补充说明宾语的,称为宾语补语,简称宾补。\\[3mm]
    补语包含了主语补语和宾语补语,表语特指主语补语,可以认为是前者的一类。\\[6mm]
    主语补语由形容词充当:This knief is \textbf{sharp}.\\[3mm]
    主语补语由名词充当:The fruit on the table is a \textbf{lemon}.\\[3mm]
    主语补语由副词充当:The aggresive debate is finally \textbf{over}.\\[3mm]



\end{document}